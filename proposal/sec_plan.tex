%!TEX root = main.tex

In this project, we plan to develop and implement a deep learning algorithm for address fingerprint classification problem.

\subsection{Feature Extraction}
%
To explore possible features, we will first apply raw fingerprint images to train a CNN for classification. The outputs of some intermediate layer  of CNN will be used as features.
%
We will also use traditional handcrafted features (\textit{e.g.},orientation filed) as inputs for CNN training. 
%
In the end, we plan to combine raw images and handcrafted features as input to train CNN.

For CNN architecture, we will first use canonical architecture ( such as 5 \textit{convolutional} + 3 \textit{fully-connected} in \textit{AlexNet}\cite{krizhevsky2012imagenet}). We will then modify the CNN architecture to improve the performance.
%

\subsection{Classifier}
%
We will consider two classifiers. The first one is the prediction layer of CNN. The values in last layer indicates the predicted probabilities of each class.
%
The second one is support vector machine (SVM).

\subsection{Data Augmentation}
%
For further improve the performance, we will use data augmentation technique to generate more train samples in order to increase the generalization.